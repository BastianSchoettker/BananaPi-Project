\chapter{Backups}
Wöchentliches Backup:
\begin{lstlisting}
#!/bin/bash
# Script fuer inkrementelles Backup mittels rsync

### Einstellungen ##
BACKUPDIR="/mnt/SSD/Backup_Weekly/"    	## Pfad zum woechentlichen Backupverzeichnis
BACKUPOLDDIR="/mnt/SSD/Backup_Monthly/" ## Pfad zum monatlichen Backupverzeichnis
DATUM="$(date +%d-%m-%Y)" ## Datumsformat einstellen
ZEIT="$(date +%H:%M)"     ## Zeitformat einstellen

### Wechsel in root damit die Pfade stimmen ##
cd /

### Backupverzeichnisse anlegen ##
mkdir -p ${BACKUPDIR}
mkdir -p ${BACKUPOLDDIR}

### Test ob Backupverzeichnis existiert und Mail an Admin bei fehlschlagen ##
if [ ! -d "${BACKUPDIR}" -a "${BACKUPOLDDIR}" ]; then

mail -s "Backupverzeichnisse nicht vorhanden!" root <<EOM
Hallo Admin,
das Backup am ${DATUM} konnte nicht erstellt werden. Das Verzeichnis ${BACKUPDIR} wurde nicht gefunden und konnte auch nicht angelegt werden.
Mit freundlichem Gruss Backupscript
EOM

 . exit 1
fi

cd /mnt/SSD/Backup_Weekly/

### Auf neue Sicherheitsupdates ueberpruefen und Verlauf speichern ##

unattended-upgrade --dry-run -d > security_update.log

### Ausfuehren des Backupprozesses wenn Installationen vorhanden sind ##

if !( grep -Fxq "InstCount=0" security_update.log ) ; then

### Update der Sicherheitspakete ##
apt-get clean
apt-get update
unattended-upgrade -d

### Nutzer auf Backupprozess hinweisen ##

clear
echo "Woechentliches Backup wird ausgefuehrt. Diesen Vorgang bitte nicht abbrechen!"
sleep 10

### Nun wird das eigentliche Backup ausgefuehrt ##
rsync -aAXv --delete --exclude={"/dev/*","/proc/*","/sys/*","/tmp/*","/run/*","/mnt/*","/media/*","/lost+found","/var/lib/ntp/ntp.drift"} / ${BACKUPDIR}


### Abfragen ob das Backup erfolgreich war ##
if [ $? -ne 0 ]; then

mail -s "Backup (${DATUM}) war fehlerhaft!" root <<EOM
Hallo Admin,
das Backup am ${DATUM} wurde mit Fehler(n) beendet.
Mit freundlichem Gruss Backupscript
EOM

else

mail -s "Backup (${DATUM}) war erfolgreich" root <<EOM
Hallo Admin,
das Backup am ${DATUM} wurde erfolgreich beendet.
Mit freundlichem Gruss Backupscript
EOM

fi
### Ende Backup Prozess ##
fi
### Loeschen der Security Log Datei - OPTIONAL ##
##rm security_update.log
exit 0
\end{lstlisting}
\newpage
Monatliches Backup:
\begin{lstlisting}
#!/bin/bash

### Wechsel in root damit die Pfade stimmen ##

cd /

### Archivieren und Kopieren des Backups in den Ordner Backups_ALT ###

cd /mnt/SSD/Backup_Monthly

touch -a /mnt/SSD/Backup.tar

clear
echo "Monatliches Backup wird ausgefuehrt. Diesen Vorgang bitte nicht abbrechen!"

tar -cpvzf Backup.tar /mnt/SSD/Backup_Weekly

echo "Monatliches Backup erfolgreich abgeschlossen."

exit 0
\end{lstlisting}

\chapter{Tmux Displaystatusanzeige}

\begin{lstlisting}
#!/bin/bash

### Aufbau des Terminals ###

tmux new-session -s 'BananaPi' -d
tmux split-window -v -t 'BananaPi'
tmux send-keys "iftop" 
tmux send-keys Enter
tmux split-window -h -t 'BananaPi'
tmux send-keys "htop" 
tmux send-keys Enter
tmux attach -t 'BananaPi'

exit 0
\end{lstlisting}
