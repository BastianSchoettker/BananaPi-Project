\section{Inbetriebnahme von FreeRADIUS2 auf OpenWRT}
\subsection{Instalation}
1. Installation des FreeRADIUS2 mit den üblichen Paketen die von verschiedenen anderen Paketen
benutzt werden:
\begin{lstlisting}
> opkg install freeradius2 freeradius2-common freeradius2-utils
Unknown package 'freeradius2'.
Unknown package 'freeradius2-common'.
Unknown package 'freeradius2-utils'.
Collected errors:
* opkg_install_cmd: Cannot install package freeradius2.
* opkg_install_cmd: Cannot install package freeradius2-common.
* opkg_install_cmd: Cannot install package freeradius2-utils.
\end{lstlisting}
-> Hat nicht funktioniert, auch nach halbstündiger Investigation konnte keine Lösung gefunden
werden, auch mit
\begin{lstlisting}
> opkg search freeradius2
> opkg info freeradius2
\end{lstlisting}
konnten keine Informationen über das Paket ermittelt werden. Schlussendlich konnten die Pakete
über das Web-Interface installiert werden, dort klappte es auf Anhieb (mit den gleichen
Bezeichnern)\\
~\\
2. Installation der FreeRADIUS2 Pakete für die MYSQL Unterstützung (+ Packet für Logging):
\begin{lstlisting}
> opkg install freeradius2-mod-sql freeradius2-mod-sql-mysql freeradius2-mod-sqllog
\end{lstlisting}
Nun sind die benötigten Pakete installiert.\\
\newpage
Die Ausgabe nach der Installation sieht folgendermaßen aus:
\begin{lstlisting}
nstalling freeradius2 (2.2.8-2) to root...
Downloading file:///etc/packages/packages/freeradius2_2.2.8-2_sunxi.ipk.
Installing freeradius2-common (2.2.8-2) to root...
Downloading file:///etc/packages/packages/freeradius2-common_2.2.8-2_sunxi.ipk.
Configuring freeradius2-common.
Configuring freeradius2.
ifconfig: br-lan: error fetching interface information: Device not found
ifconfig: br-lan: error fetching interface information: Device not found
radiusd: Invalid IP Address or hostname "-p"
\end{lstlisting}
Mit br-lan ist hier das sogenannte „bridged lan virtual interface“ gemeint (liegt daran dass es
standardmäßig so vorkonfiguriert ist).\\
Beim Versuch FreeRADIUS2 im Debugging-Modus zu starten kommt:\
\begin{lstlisting}
> radiusd -X
radiusd: FreeRADIUS Version 2.2.8, for host arm-openwrt-linux-gnu, built on Oct 3 2016 at 22:22:30
Copyright (C) 1999-2015 The FreeRADIUS server project and contributors.
There is NO warranty; not even for MERCHANTABILITY or FITNESS FOR A PARTICULAR PURPOSE.
You may redistribute copies of FreeRADIUS under the terms of the GNU General Public License.
For more information about these matters, see the file named COPYRIGHT.
Starting - reading configuration files ...
including configuration file /etc/freeradius2/radiusd.conf
including configuration file /etc/freeradius2/clients.conf
including files in directory /etc/freeradius2/modules/
including configuration file /etc/freeradius2/eap.conf
Unable to open file "/etc/freeradius2/eap.conf": No such file or directory
Errors reading or parsing /etc/freeradius2/radiusd.conf
\end{lstlisting}
EAP ist das Extensible Authentication Protocol, ein allgemeines Authentifizierungsprotokoll. Also
wurden die noch benötigten FreeRADIUS2 Abhängigkeiten Installiert:
\begin{lstlisting}
> opkg install freeradius2-mod-chap freeradius2-mod-detail freeradius2-mod-eap freeradius2-mod-eap-md5 freeradius2-mod-eap-mschapv2 freeradius2-mod-eap-peap freeradius2-mod-eap-tls freeradius2-mod-eap-ttls
freeradius2-mod-exec freeradius2-mod-files freeradius2-mod-logintime
freeradius2-mod-mschap freeradius2-mod-pap freeradius2-mod-passwd
freeradius2-mod-preprocess freeradius2-mod-radutmp
\end{lstlisting}
3. Installation des MYSQL-Servers
\begin{lstlisting}
> opkg install mysql-server
Installing mysql-server (5.1.73-1) to root...
Downloading file:///etc/packages/packages/mysql-server_5.1.73-1_sunxi.ipk.
Installing libmysqlclient (5.1.73-1) to root...
Downloading file:///etc/packages/packages/libmysqlclient_5.1.73-1_sunxi.ipk.
Configuring libmysqlclient.
Configuring mysql-server.
/etc/init.d/mysqld: Error: datadir '/mnt/data/mysql/' in /etc/my.cnf
doesn't exist
\end{lstlisting}
Nach einiger Zeit stieß ich im Internet auf vergleichbare Probleme und es gab einen Work-Around
mit dem der MYSQL-Server sich schließlich doch installieren lies.\\
~\\
4. Einrichtung einer Zertifikatskette für das EAP
\begin{lstlisting}
>opkg install openssl-util
\end{lstlisting}
->Erstellung der Zertifikatskette mit OpenSSL (...)\\
\newpage
5. Bevor man allerdings mit der Konfiguration beginnt, sollte man jedoch den Radius-Daemon
stoppen: Über das Web-Interface LuCi unter dem Menüpunkt „System“ ->„Startup“ wird der
radiusd deaktiviert („disable“) und erstmal vom automatischen Starten beim Booten
ausgenommen. Für Test- und Debuggingzwecke empfiehlt es sich außerdem, den Radius Server
umzukonfigurieren, sodass dieser auch auch auf Localhost horcht und nicht nur auf die IP Adresse
im Netzwerk. Dafür editiert man die Datei
\begin{lstlisting}
> vim /etc/init.d/radiusd
\end{lstlisting}
und ersetzt die Zeile 17
\begin{lstlisting}
radiusd -i $IPADDR -p 1812,1813 $OPTIONS
\end{lstlisting}
durch
\begin{lstlisting}
radiusd $OPTIONS
\end{lstlisting}






















