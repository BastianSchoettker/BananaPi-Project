\chapter{Projektübersicht}

\section{Kontext}
Das Projekt mit dem Titel Router mit embedded Board Banana Pi R1 wird als Se-
mesterprojekt im Sommersemester 2017 an der Hochschule Furtwangen durch-
geführt. Das Projekt wurde von Dr. Jiri Spale ins Leben gerufen und wird
intern, ohne die Kooperation mit einem Unternehmen durchgeführt. Die Anzahl der studentischen Projektteilnehmer beträgt fünf.

\section{Ziel}
Das primäre Ziel des Projektes ist es, das bestehende Router-Projekt weiterzuentwickeln. Folgende Funktionalitäten sollen implementiert werden:\\
~\\
\begin{itemize}
\item Implementierung eines RADIUS-Servers zur zentralen Authentifizierung von Anwendern
\item Implementierung einse Voucher-Systems
\item Installation des Betriebssystems IP Fire
\item Mehrere WLAN Access Points auf einem WLAN-Chip
\item Mail-Server zum Statusbericht der Backups
\item SAMBA/NFS Server
\item Automatische Aktuatisierung von Programmpaketen des Systems
\item Implementierung einer Display-Statusanzeige
\end{itemize}
~\\
Die Funktionen sollen auf dem Betriebssystem “Bananian”, ein auf das Banana Pi zugeschnittenes Debian, implementiert werden.
