\chapter{Betriebssysteme\"ubersicht}

\section{OpenWRT}

\section{IPFire}
\begin{wrapfigure}{r}{5cm}
\centering
\includegraphics[width=3.5cm]{pictures/Jakob/IPFire}
\caption{IPFire Logo}
Quelle: \cite{fire1}
\end{wrapfigure}
Das Betriebssystem IPFire ist darauf ausgelegt, als Router genutzt zu werden, wobei das Augenmerkauf der implementierung einer Firewall oder eines VPN Gateways gelegt ist.
Das vorgehende Projekt hatte dieses Betriebssystem als Alternative vorgeschlagen, washalb wir einen Blick darauf geworfen haben. Zwei Abbilder für den Betrieb auf Arm Prozessoren stehen auf der Webseite des Betriebssystems bereit. \cite{fire} \\
Bei beide Abbildern stellte sich allerdings herraus, dass die HDMI Schnittstelle nicht unterschtützt wird.
Auch eine Verbindung SSH nicht möglich ist, da IPFire auf eine Konfiguration über das Webinterface ausgelegt ist.
Eines der beiden Abbilder ist auf die Konfiguration über eine Serielle Schnittstelle Ausgelegt. Da uns aber für diese Schnittstelle keine Wekzeuge und Adapter bereitstehen, haben wir auch von dieser Option abgelassen.\\

\section{Bananian}
\begin{wrapfigure}{r}{5cm}
\centering
\includegraphics[width=3cm]{pictures/Jakob/Bananian}
\caption{Bananian Logo}
Quelle: \cite{bananian1}
\end{wrapfigure}
Der Banana Pi R1 wird von dem Betribssystem Bananian vollständig unterstützt. Dieses Betriebssystem basiert auf Debian 8 und nutzt das Debian Jessie armhf repositorie. \cite{bananian}\\
Neben OpenWRT wurde das Projekt mit Bananian gestartet. Da die Weiterentwicklung von Bananian am 2.4.2017 eingestellt wurde haben wir uns einschieden auf das Betriebssystem Armbian zu wechseln. \cite{bananian2}








\section{Armbian}
