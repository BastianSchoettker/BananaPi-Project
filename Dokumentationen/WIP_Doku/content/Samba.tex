\section{Samba}
Um den Dateizugriff zu erleichtern, wurde ein Samba-Server auf dem Pi implementiert.\\
Samba ermöglicht es von nahezu jedem Gerät auf ein Freigegebenes Verzeichnis auf dem Server zuzugreifen. Voraussetzung ist, dass das Client Betriebssystem das SMB-Protokoll unterstützt.\\
Die meisten modernen Betriebssysteme, wie Windows, MacOS und andere Unixoide besitzen Samba Funktionalität.\\
~\\
Einrichtung\\
Voraussetzungen:\\
- Internetzugriff\\
- Texteditor (Nano)\\
- SSH/Physikalischen Zugriff\\
\\
1. Samba installieren:
\begin{lstlisting}
sudo apt-get update
sudo apt-get install samba
\end{lstlisting}
~\\
2. Benutzer für Samba erstellen (Hat keinen Shell-Zugriff):
\begin{lstlisting}
useradd sambausr --shell /bin/false
\end{lstlisting}
~\\
3. Passwort für den Benutzer in Samba setzen:
\begin{lstlisting}
smbpasswd -a <user_name>
\end{lstlisting}
~\\
4. Verzeichnis im Homer erstellen:
\begin{lstlisting}
mkdir /home/sambausr
mkdir /home/sambausr/samba
\end{lstlisting}
~\\
5. Berechtigungen setzen:
\begin{lstlisting}
chown sambausr:sambausr /home/sambausr/
chown sambausr:sambausr /home/sambausr/samba/
\end{lstlisting}
~\\
6. Backup der Samba Konfiguration im Homeverzeichnis machen:
\begin{lstlisting}
cp /etc/samba/smb.conf ~
\end{lstlisting}
~\\
7. Config bearbeiten:
\begin{lstlisting}
nano /etc/samba/smb.conf
\end{lstlisting}
~\\
7.1 folgendes am Ende der Konfiguration Einfügen:
\begin{lstlisting}
[samba]
path = /home/sambausr/samba
valid users = sambausr
read only = no
\end{lstlisting}
~\\
8. Service neustarten:
\begin{lstlisting}
service smbd restart
\end{lstlisting}
~\\
9. Config testen:
\begin{lstlisting}
testparm
\end{lstlisting}
~\\Quelle: https://help.ubuntu.com/community/How to Create a Network Share... \cite{samba}
